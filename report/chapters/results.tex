\section{Conclusions}
\label{results}

This last Section shows the results obtained by the model found 
on the previous Section on the different test sets and
gives an overview about possible future works to be made.

\subsection{Test set results}
As stated on the first Section, the five test sets are made out of the 
folds number five, seven, eight, nine and ten.
Evaluating the model found by the random search on the test sets 
gives the following results: 

\begin{center}
    \begin{tabular}{ |l|r| } 
        \hline
        Test set & Accuracy\\
        \hline
        Fold 5 & 0.6496 \\
        Fold 7 & 0.6169 \\
        Fold 8 & 0.7395 \\
        Fold 9 & 0.6789 \\
        Fold 10 & 0.6774 \\ 
        \hline
    \end{tabular}
\end{center}

Finally, the mean accuracy and standard deviation are: 
\begin{center}
    \begin{tabular}{ |r|r| } 
        \hline
        Mean accuracy & Standard deviation\\
        \hline
        0.6724 & 0.0403 \\
        \hline
    \end{tabular}
\end{center}

\subsection{Future works}

Results on the test sets are promising but there is definitely 
room for improvement.

Indeed, a more refined training set creation can be made, by exploiting 
more features from the Librosa library, testing different type 
of scalers and experimenting with different feature selection 
techniques.

Finally, the models used in the project are simple. More complex 
neural networks, for instance with convolutional layers, could 
perform better, also, a more extensive random search might improve
results.