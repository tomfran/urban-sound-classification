\section{Introduction}

The goal of this project is to build a neural network to classify audio 
files from the \emph{UrbanSound8k} dataset~\cite{dataset}.

This dataset contains audio divided in ten classes, each one representing 
a different type of city sound, for instance, we can find \emph{car horns}, 
\emph{dogs barking}, \emph{sirens}, etc. A deeper discussion about 
the dataset is made at Subsection \vref*{dataset-structure}.

The presented methodology is composed of three main parts.
The first step is to extract relevant features from audio files, 
this is discussed on Section \vref*{feature-extraction}.\\
The next step consists in composing and refining a neural network 
to classify the data obtained from the feature extraction phase. This part is
discussed in Section \vref*{model-definition}.\\
Lastly, results from the classification, namely accuracy 
and standard deviation among test sets, and possible future works 
are presented in Section \vref*{results}.

\paragraph{Project structure}

The project folder is structured as follows:
\begin{itemize}
    \item \emph{src}: this contains the source code for the
    project. Each sub-folder contains code for a specific part of the 
    processing. In particular, the \emph{data} folder holds classes to 
    extract features and to manage the dataset, the \emph{model} folder
    contains the class to create the Neural Network, then \emph{utils} 
    stores utility functions to measure performances;
    \item \emph{data}: here data is stored, there is a \emph{processed}
    and \emph{raw} sub-folders, where the first stores computed datasets 
    and the latter original data;
    \item \emph{models}: trained models are saved here;
    \item \emph{notebooks}: the folder contains the Jupyter notebooks 
    used in the project, where code from src executed.
\end{itemize}
The code is written in Python.
\newpage